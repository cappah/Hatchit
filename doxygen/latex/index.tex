\href{https://gitter.im/thirddegree/Hatchit?utm_source=badge&utm_medium=badge&utm_campaign=pr-badge&utm_content=badge}{\tt } \href{doxygen/html/index.html}{\tt }

An Open Source 3D Game Engine written in C++ focusing on support for Vulkan 



\subsubsection*{Contributing}

Any issues not related to the build system should go onto the page of the repo best suited for that issue. Graphics code goes onto the Hatchit\+Graphics page for instance.

If you would like to submit a Pull Request it\textquotesingle{}s probably best to clone this repo and follow the build instructions to get a working copy. Then fork the repo that you\textquotesingle{}d like to contribute code to and modify your .gitconfig file to point to your fork. Then you can edit, test and push all you\textquotesingle{}d like before submitting a pull request.

\subsubsection*{Build instructions}

This is subject to change and may not be complete or entirely accurate!

\paragraph*{\hyperlink{namespaceLinux}{Linux}}

All testing and development so far has been done on Ubuntu 16.\+04 X\+E\+N\+I\+AL

Install the following dependencies\+:
\begin{DoxyItemize}
\item G\+CC (minimum required version. 5.\+0) -- for C++11/14 support
\item C\+Make
\item G\+L\+F\+W3
\item Bullet3
\item Assimp
\item Tiny\+X\+M\+L2
\item Python 3.\+5 (with pip and virtualenv)
\item Vulkan S\+DK (For Vulkan Support)
\end{DoxyItemize}

Then follow these simple directions\+:
\begin{DoxyItemize}
\item Clone down the repo with {\ttfamily git clone -\/-\/recursive \href{http://github.com/thirddegree/Hatchit}{\tt http\+://github.\+com/thirddegree/\+Hatchit}}
\item Make a build dir (e.\+g. mkdir build)
\item cd into build/ and run cmake ..
\item Now just run make to build \hyperlink{namespaceHatchit}{Hatchit}
\end{DoxyItemize}

\paragraph*{\hyperlink{namespaceWindows}{Windows}}

This is a bit more of a pain. All dependencies are submodules in the Third Party directory and will need to be built before you build \hyperlink{namespaceHatchit}{Hatchit}.

All testing and development has been done on \hyperlink{namespaceWindows}{Windows} 10 with Visual Studio 2015

\subparagraph*{Pre-\/\+Build}

We recommend using some sort of cmd replacement in \hyperlink{namespaceWindows}{Windows} such as cmder or some sort of bash shell
\begin{DoxyItemize}
\item Install C\+Make and make sure it is in your path
\item Make sure that you can run {\ttfamily msbuild.\+exe} from your shell. If it\textquotesingle{}s not there try running the {\ttfamily vcvarsall.\+bat} file located at {\ttfamily C\+:\textbackslash{}Program Files (x86)\textbackslash{}Microsoft Visual Studio 14.\+0\textbackslash{}VC\textbackslash{}vcvarsall.\+bat} for Visual Studio 2015
\item Clone down the repo with {\ttfamily git clone -\/-\/recursive \href{http://github.com/thirddegree/Hatchit}{\tt http\+://github.\+com/thirddegree/\+Hatchit}}
\end{DoxyItemize}

To build with Vulkan support you M\+U\+ST have installed\+:
\begin{DoxyItemize}
\item Vulkan S\+DK
\item Vulkan supported drivers!
\end{DoxyItemize}

The build system requires that you have a working Python 3.\+5 installation.

\subparagraph*{Third Party}

Next up is building all the dependencies. This should only have to be done once for your machine. After this you won\textquotesingle{}t have to worry about {\ttfamily vcvarsall.\+bat} but you will need C\+Make.


\begin{DoxyItemize}
\item Run the {\ttfamily setup.\+bat} file. This should configure A\+ND B\+U\+I\+LD all your dependencies
\end{DoxyItemize}

\subparagraph*{\hyperlink{namespaceHatchit}{Hatchit}}

This is the easy part!


\begin{DoxyItemize}
\item {\ttfamily cd} into {\ttfamily build}
\item Open the Hatchit.\+sln solution file or execute it with {\ttfamily M\+S\+Build.\+exe} to build all targets 
\end{DoxyItemize}